\documentclass{mn2e}
\usepackage{footnote}
\usepackage{natbib}

\begin{document}
\title[Conference abstract]{Galaxy Zoo 2: Investigating the Star Formation History of the Green Valley}
\author[Smethurst et al. 2014]{R. ~J. ~Smethurst,$^1$ C. ~J. ~Lintott,$^{1,2}$ B. ~D. ~Simmons,$^{1}$ \\ $^1$ Oxford Astrophysics, Department of Physics, University of Oxford, Keble Road, Oxford, OX1 3RH, UK \\ $^2$ Adler Planetarium, 1300 S Lake Shore Drive, Chicago, IL, 60605, USA }

\maketitle

\begin{abstract}
Does galactic evolution proceed through the Green Valley via multiple pathways or as a single population? Motivated by recent results which used a toy model to highlight radically different evolutionary pathways between early- and late-type galaxies, we present results from an advanced Bayesian approach to this problem wherein we model the star formation history of a galaxy and compare the predicted and observed optical and near-ultraviolet colours in the model parameter space. We investigate the most probable values for these parameters for both disc-like and elliptical-like populations of galaxies, incorporating the morphological vote fractions from Galaxy Zoo\footnotemark[1] into our analysis. We will discuss the implications of our results on the understanding of the simultaneous morphological-colour evolution of galaxies, particularly of those residing in the Green Valley. 
\end{abstract}

\footnotetext[1]{This investigation has been made possible by the participation of more than 250,000 volunteers in the Galaxy Zoo project. Their contributions are individually acknowledged at http://www.galaxyzoo.org/volunteers.aspx}

\end{document}
